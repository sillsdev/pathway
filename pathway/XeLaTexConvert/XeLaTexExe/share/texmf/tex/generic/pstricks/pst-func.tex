%%
%% This is file `pst-func.tex',
%%
%% IMPORTANT NOTICE:
%%
%% Package `pst-func.tex'
%%
%% Herbert Voss <voss@pstricks.de>
%%
%% This program can be redistributed and/or modified under the terms
%% of the LaTeX Project Public License Distributed from CTAN archives
%% in directory macros/latex/base/lppl.txt.
%%
%% DESCRIPTION:
%%   `pst-func' is a PSTricks package to plot special functions
%%
%% For a ChangeLog go the the end
%%
\csname PSTfuncLoaded\endcsname
\let\PSTfuncLoaded\endinput
% Requires PSTricks, pst-node
\ifx\PSTricksLoaded\endinput\else\input pstricks.tex\fi
\ifx\PSTnodesLoaded\endinput\else\input pst-plot.tex\fi
\ifx\PSTXKeyLoaded\endinput\else\input pst-xkey.tex \fi
%
\edef\PstAtCode{\the\catcode`\@} \catcode`\@=11\relax
% interface to the `xkeyval' package
\pst@addfams{pst-func}

\def\fileversion{0.40}
\def\filedate{2005/04/09}
\message{`PST-func' v\fileversion, \filedate\space (Herbert Voss)}
%
\pstheader{pst-func.pro}
%\def\pst@funcdict{tx@FuncDict begin }
%\def\tx@saveCoor{\pst@3ddict saveCoor end }
%\def\tx@ConvertTo2D{\pst@3ddict ConvertTo2D end }
%
\define@key[psset]{pst-func}{xShift}{\def\psk@xShift{#1}}
\psset[pst-func]{xShift=0}
%
\define@key[psset]{pst-func}{cosCoeff}{\def\psk@cosCoeff{#1}}
\define@key[psset]{pst-func}{sinCoeff}{\def\psk@sinCoeff{#1}}
\psset[pst-func]{cosCoeff=0,sinCoeff=1} % coeff=a0 a1 a2 a3 ...
%
\def\psFourier{\@ifnextchar[{\psFourier@i}{\psFourier@i[]}}
\def\psFourier@i[#1]#2#3{{%
  \pst@killglue
  \psset{#1}
  \psplot{#2}{#3}{%
      /type (cos) def
      /Fourier {
        aload length /n exch def
        n -1 roll 2 div n 1 roll % a0/2
        n 1 sub -1 0 {
          /i exch def
          i x mul 180 mul 3.141592 div
          type (sin) eq {sin}{cos} ifelse
          mul n 1 roll
        } for
        n 1 sub -1 1 { pop add } for
      } def
      [\psk@cosCoeff] Fourier
      /type (sin) def
      [0 \psk@sinCoeff] Fourier add
    }%
}\ignorespaces}
%
\define@key[psset]{pst-func}{coeff}{\def\psk@coeff{#1}}
\define@key[psset]{pst-func}{Abbreviation}{\def\psk@Deriviation{#1}}% compatibility
\define@key[psset]{pst-func}{Derivation}{\def\psk@Derivation{#1}}
\newif\ifPst@markZeros%	
\define@key[psset]{pst-func}{markZeros}[true]{\@nameuse{Pst@markZeros#1}}
\define@key[psset]{pst-func}{epsZero}{\def\psk@epsZero{#1}}
\define@key[psset]{pst-func}{dZero}{\def\psk@dZero{#1}}
\define@key[psset]{pst-func}{zeroLineTo}{\def\psk@zeroLineTo{#1}}
\define@key[psset]{pst-func}{zeroLineColor}{\pst@getcolor{#1}\psk@zeroLineColor}
\newdimen\psk@zeroLineWidth
\define@key[psset]{pst-func}{zeroLineWidth}{\pssetlength\psk@zeroLineWidth{#1}}
\define@key[psset]{pst-func}{zeroLineStyle}{%
  \@ifundefined{psls@#1}%
    {\@pstrickserr{Line style `#1' not defined}\@eha}%
    {\edef\psk@zeroLineStyle{#1}}%
}
\psset[pst-func]{%
       coeff=0 1,      % coeff=a0 a1 a2 a3 ...
       Derivation=0, % 0 is the original function
       markZeros=false,% no dots for the zeros
       epsZero=0.1,    % the distance between two zero points   
       dZero=0.1,      % the distance of the x value for scanning the function
       zeroLineTo=-1,  % a line to the value of the lineTo's Derivation (-1= none)
       zeroLineStyle=dashed,%
       zeroLineWidth=0.5\pslinewidth,%
       zeroLineColor=black}%
%
\def\psPolynomial{\pst@object{psPolynomial}}
\def\psPolynomial@i#1#2{{%
  \begin@OpenObj
  \@nameuse{beginplot@\psplotstyle}%
  \gdef\psplot@init{}%
  \@nameuse{testqp@\psplotstyle}%
  \addto@pscode{%
    tx@FuncDict begin
    /coeff [ \psk@coeff ] def
    /x0 #1 def /x1 #2 def
    /dx x1 x0 sub \psk@plotpoints\space div def
    /Derivation \psk@Derivation\space def
    \ifPst@markZeros 
      gsave
      \pst@number\psk@zeroLineWidth SLW
      \pst@usecolor\psk@zeroLineColor
      \psk@epsZero\space \psk@dZero\space FindZeros 
      pstZeros aload length {
        /xZero exch def
        xZero \pst@number\psxunit mul /xPixel exch def
        \psk@dotsize
        \@nameuse{psds@\psk@dotstyle}%
        xPixel 0 Dot 
        \psk@zeroLineTo\space 0 ge { % line to function \psk@lineTo
          xPixel 0 moveto
          xZero coeff \psk@zeroLineTo\space FuncValue 
          \pst@number\psyunit mul xPixel exch L
          \@nameuse{psls@\psk@zeroLineStyle}
        } if
      } repeat
      grestore
    \fi
    /x x0 def
    /xy {
      x \psk@xShift\space sub coeff Derivation FuncValue \pst@number\psyunit mul 
      x \pst@number\psxunit mul exch
    } def
    xy moveto
  }%
  \if@pst%	lines and dots
    \psPolynomial@ii%
  \else%	curves
    \psPolynomial@iii%
  \fi%
  \end@OpenObj
}\ignorespaces}
%
\def\psPolynomial@ii{%
  \addto@pscode{%
     xy \@nameuse{beginqp@\psplotstyle}
    \psk@plotpoints {
      xy \@nameuse{doqp@\psplotstyle}
      /x x dx add def
    } repeat
    xy \@nameuse{doqp@\psplotstyle}
    end
  }%
  \@nameuse{endqp@\psplotstyle}%
}
\def\psPolynomial@iii{%    curves
  \addto@pscode{%
    mark
    /n 2 def
    \psk@plotpoints {
      xy 
      n 2 roll
      /n n 2 add def
      /x x dx add def
    } repeat
    /x x1 def
    xy
    n 2 roll
    end
  }%
  \@nameuse{endplot@\psplotstyle}%
}
%
% Bessel 2004-06-08
% Manuel Luque, Herbert Voss
% Look at the end for some more documentation about the algorithm
%
\define@key[psset]{pst-func}{constI}{\edef\psk@constI{#1}}
\define@key[psset]{pst-func}{constII}{\edef\psk@constII{#1}}
\psset{constI=1,constII=0}
%
\def\psBessel{\@ifnextchar[{\psBessel@i}{\psBessel@i[]}}
\def\psBessel@i[#1]#2#3#4{{%%%  #2 = n
  \pst@killglue
  \psset{plotpoints=500}%
  \psset{#1}%
  \parametricplot{#3}{#4}{%
    /J1 0 def
    /k { 57.29577951 mul } def
    /xBessel t k def
    0 0.1 180 {
      /tB exch k def
      /J1 J1 0.1 xBessel
        tB sin mul tB #2\space mul sub cos mul add def
    } for
    t J1 180 div \psk@constI\space mul \psk@constII\space add
  }%
}\ignorespaces}
%
\define@key[psset]{pst-func}{sigma}{\def\psk@sigma{#1}}
\psset[pst-func]{sigma=0.5}
%
\def\psGauss{\@ifnextchar[{\psGauss@i}{\psGauss@i[]}}
\def\psGauss@i[#1]#2#3{{%
    \pst@killglue%
    \psset{plotpoints=200}%
    \psset{#1}%
    \pstVerb{%
        /euler 2.718282 def
        /Const 1 \psk@sigma\space div 6.2831 sqrt div def
	/x0 \psk@xShift\space def
    }%
    \psplot{#2}{#3}{%
        euler x x0 sub dup mul 2 div \psk@sigma\space dup mul div neg exp Const mul%
    }%
}\ignorespaces}
%
\catcode`\@=\PstAtCode\relax
%
%% END: pst-abspos.tex
\endinput
%
